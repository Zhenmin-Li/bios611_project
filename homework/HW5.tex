\PassOptionsToPackage{unicode=true}{hyperref} % options for packages loaded elsewhere
\PassOptionsToPackage{hyphens}{url}
%
\documentclass[]{article}
\usepackage{lmodern}
\usepackage{amssymb,amsmath}
\usepackage{ifxetex,ifluatex}
\usepackage{fixltx2e} % provides \textsubscript
\ifnum 0\ifxetex 1\fi\ifluatex 1\fi=0 % if pdftex
  \usepackage[T1]{fontenc}
  \usepackage[utf8]{inputenc}
  \usepackage{textcomp} % provides euro and other symbols
\else % if luatex or xelatex
  \usepackage{unicode-math}
  \defaultfontfeatures{Ligatures=TeX,Scale=MatchLowercase}
\fi
% use upquote if available, for straight quotes in verbatim environments
\IfFileExists{upquote.sty}{\usepackage{upquote}}{}
% use microtype if available
\IfFileExists{microtype.sty}{%
\usepackage[]{microtype}
\UseMicrotypeSet[protrusion]{basicmath} % disable protrusion for tt fonts
}{}
\IfFileExists{parskip.sty}{%
\usepackage{parskip}
}{% else
\setlength{\parindent}{0pt}
\setlength{\parskip}{6pt plus 2pt minus 1pt}
}
\usepackage{hyperref}
\hypersetup{
            pdftitle={HW5},
            pdfauthor={Zhenmin Li},
            pdfborder={0 0 0},
            breaklinks=true}
\urlstyle{same}  % don't use monospace font for urls
\usepackage[margin=1in]{geometry}
\usepackage{color}
\usepackage{fancyvrb}
\newcommand{\VerbBar}{|}
\newcommand{\VERB}{\Verb[commandchars=\\\{\}]}
\DefineVerbatimEnvironment{Highlighting}{Verbatim}{commandchars=\\\{\}}
% Add ',fontsize=\small' for more characters per line
\usepackage{framed}
\definecolor{shadecolor}{RGB}{248,248,248}
\newenvironment{Shaded}{\begin{snugshade}}{\end{snugshade}}
\newcommand{\AlertTok}[1]{\textcolor[rgb]{0.94,0.16,0.16}{#1}}
\newcommand{\AnnotationTok}[1]{\textcolor[rgb]{0.56,0.35,0.01}{\textbf{\textit{#1}}}}
\newcommand{\AttributeTok}[1]{\textcolor[rgb]{0.77,0.63,0.00}{#1}}
\newcommand{\BaseNTok}[1]{\textcolor[rgb]{0.00,0.00,0.81}{#1}}
\newcommand{\BuiltInTok}[1]{#1}
\newcommand{\CharTok}[1]{\textcolor[rgb]{0.31,0.60,0.02}{#1}}
\newcommand{\CommentTok}[1]{\textcolor[rgb]{0.56,0.35,0.01}{\textit{#1}}}
\newcommand{\CommentVarTok}[1]{\textcolor[rgb]{0.56,0.35,0.01}{\textbf{\textit{#1}}}}
\newcommand{\ConstantTok}[1]{\textcolor[rgb]{0.00,0.00,0.00}{#1}}
\newcommand{\ControlFlowTok}[1]{\textcolor[rgb]{0.13,0.29,0.53}{\textbf{#1}}}
\newcommand{\DataTypeTok}[1]{\textcolor[rgb]{0.13,0.29,0.53}{#1}}
\newcommand{\DecValTok}[1]{\textcolor[rgb]{0.00,0.00,0.81}{#1}}
\newcommand{\DocumentationTok}[1]{\textcolor[rgb]{0.56,0.35,0.01}{\textbf{\textit{#1}}}}
\newcommand{\ErrorTok}[1]{\textcolor[rgb]{0.64,0.00,0.00}{\textbf{#1}}}
\newcommand{\ExtensionTok}[1]{#1}
\newcommand{\FloatTok}[1]{\textcolor[rgb]{0.00,0.00,0.81}{#1}}
\newcommand{\FunctionTok}[1]{\textcolor[rgb]{0.00,0.00,0.00}{#1}}
\newcommand{\ImportTok}[1]{#1}
\newcommand{\InformationTok}[1]{\textcolor[rgb]{0.56,0.35,0.01}{\textbf{\textit{#1}}}}
\newcommand{\KeywordTok}[1]{\textcolor[rgb]{0.13,0.29,0.53}{\textbf{#1}}}
\newcommand{\NormalTok}[1]{#1}
\newcommand{\OperatorTok}[1]{\textcolor[rgb]{0.81,0.36,0.00}{\textbf{#1}}}
\newcommand{\OtherTok}[1]{\textcolor[rgb]{0.56,0.35,0.01}{#1}}
\newcommand{\PreprocessorTok}[1]{\textcolor[rgb]{0.56,0.35,0.01}{\textit{#1}}}
\newcommand{\RegionMarkerTok}[1]{#1}
\newcommand{\SpecialCharTok}[1]{\textcolor[rgb]{0.00,0.00,0.00}{#1}}
\newcommand{\SpecialStringTok}[1]{\textcolor[rgb]{0.31,0.60,0.02}{#1}}
\newcommand{\StringTok}[1]{\textcolor[rgb]{0.31,0.60,0.02}{#1}}
\newcommand{\VariableTok}[1]{\textcolor[rgb]{0.00,0.00,0.00}{#1}}
\newcommand{\VerbatimStringTok}[1]{\textcolor[rgb]{0.31,0.60,0.02}{#1}}
\newcommand{\WarningTok}[1]{\textcolor[rgb]{0.56,0.35,0.01}{\textbf{\textit{#1}}}}
\usepackage{graphicx,grffile}
\makeatletter
\def\maxwidth{\ifdim\Gin@nat@width>\linewidth\linewidth\else\Gin@nat@width\fi}
\def\maxheight{\ifdim\Gin@nat@height>\textheight\textheight\else\Gin@nat@height\fi}
\makeatother
% Scale images if necessary, so that they will not overflow the page
% margins by default, and it is still possible to overwrite the defaults
% using explicit options in \includegraphics[width, height, ...]{}
\setkeys{Gin}{width=\maxwidth,height=\maxheight,keepaspectratio}
\setlength{\emergencystretch}{3em}  % prevent overfull lines
\providecommand{\tightlist}{%
  \setlength{\itemsep}{0pt}\setlength{\parskip}{0pt}}
\setcounter{secnumdepth}{0}
% Redefines (sub)paragraphs to behave more like sections
\ifx\paragraph\undefined\else
\let\oldparagraph\paragraph
\renewcommand{\paragraph}[1]{\oldparagraph{#1}\mbox{}}
\fi
\ifx\subparagraph\undefined\else
\let\oldsubparagraph\subparagraph
\renewcommand{\subparagraph}[1]{\oldsubparagraph{#1}\mbox{}}
\fi

% set default figure placement to htbp
\makeatletter
\def\fps@figure{htbp}
\makeatother


\title{HW5}
\author{Zhenmin Li}
\date{2020/10/24}

\begin{document}
\maketitle

\hypertarget{q1}{%
\section{Q1}\label{q1}}

In the model trained using dataset in the previous homework (PRE),
weight and height has a similar importance on the prediction. In the
model trained using dataset in the current homework (CURR), the
prediction result almost fully depends on the weight.

The result from the model based on PRE mostly cumulated on around 0.5,
which means the model is not quiet sure about the results. The one from
CURR mostly cumulated on about 0 and 1, which means the model is quite
sure for most predictions.

I think this is because CURR is not normalized. According to our common
sense, most males have higher weights than females, and the model would
predict by weight.

\begin{Shaded}
\begin{Highlighting}[]
\KeywordTok{library}\NormalTok{(dplyr)}
\end{Highlighting}
\end{Shaded}

\begin{verbatim}
## 
## Attaching package: 'dplyr'
\end{verbatim}

\begin{verbatim}
## The following objects are masked from 'package:stats':
## 
##     filter, lag
\end{verbatim}

\begin{verbatim}
## The following objects are masked from 'package:base':
## 
##     intersect, setdiff, setequal, union
\end{verbatim}

\begin{Shaded}
\begin{Highlighting}[]
\KeywordTok{library}\NormalTok{(gbm)}
\end{Highlighting}
\end{Shaded}

\begin{verbatim}
## Loaded gbm 2.1.8
\end{verbatim}

\begin{Shaded}
\begin{Highlighting}[]
\KeywordTok{library}\NormalTok{(tidyverse)}
\end{Highlighting}
\end{Shaded}

\begin{verbatim}
## -- Attaching packages ------------------------------------------------------------------------------- tidyverse 1.3.0 --
\end{verbatim}

\begin{verbatim}
## v ggplot2 3.3.2     v purrr   0.3.4
## v tibble  3.0.4     v stringr 1.4.0
## v tidyr   1.1.2     v forcats 0.5.0
## v readr   1.4.0
\end{verbatim}

\begin{verbatim}
## -- Conflicts ---------------------------------------------------------------------------------- tidyverse_conflicts() --
## x dplyr::filter() masks stats::filter()
## x dplyr::lag()    masks stats::lag()
\end{verbatim}

\begin{Shaded}
\begin{Highlighting}[]
\KeywordTok{library}\NormalTok{(ggplot2)}
\KeywordTok{library}\NormalTok{(gridExtra)}
\end{Highlighting}
\end{Shaded}

\begin{verbatim}
## 
## Attaching package: 'gridExtra'
\end{verbatim}

\begin{verbatim}
## The following object is masked from 'package:dplyr':
## 
##     combine
\end{verbatim}

\begin{Shaded}
\begin{Highlighting}[]
\KeywordTok{library}\NormalTok{(ggfortify)}
\KeywordTok{library}\NormalTok{(factoextra)}
\end{Highlighting}
\end{Shaded}

\begin{verbatim}
## Welcome! Want to learn more? See two factoextra-related books at https://goo.gl/ve3WBa
\end{verbatim}

\begin{Shaded}
\begin{Highlighting}[]
\NormalTok{nice_names <-}\StringTok{ }\ControlFlowTok{function}\NormalTok{(df)\{}
  \KeywordTok{names}\NormalTok{(df) <-}\StringTok{ }\KeywordTok{names}\NormalTok{(df) }\OperatorTok\StringTok{ }\KeywordTok{str_replace_all}\NormalTok{(}\StringTok{"[^a-zA-Z0-9]+"}\NormalTok{,}\StringTok{"_"}\NormalTok{) }\OperatorTok
\StringTok{    }\KeywordTok{str_replace_all}\NormalTok{(}\StringTok{"[_]+$"}\NormalTok{,}\StringTok{""}\NormalTok{) }\OperatorTok
\StringTok{    }\KeywordTok{str_replace_all}\NormalTok{(}\StringTok{"^[_]+"}\NormalTok{,}\StringTok{""}\NormalTok{) }\OperatorTok
\StringTok{    }\KeywordTok{tolower}\NormalTok{();}
\NormalTok{  df}
\NormalTok{\};}

\NormalTok{info <-}\StringTok{ }\NormalTok{timetk}\OperatorTok{::}\KeywordTok{tk_tbl}\NormalTok{(data.table}\OperatorTok{::}\KeywordTok{fread}\NormalTok{(}\StringTok{"500_Person_Gender_Height_Weight_Index.csv"}\NormalTok{, }\DataTypeTok{header=}\NormalTok{T, }\DataTypeTok{stringsAsFactors=}\NormalTok{T)) }\OperatorTok
\StringTok{  }\KeywordTok{drop_na}\NormalTok{() }\OperatorTok\StringTok{ }
\StringTok{  }\KeywordTok{nice_names}\NormalTok{() }\OperatorTok
\StringTok{  }\KeywordTok{mutate}\NormalTok{(}\DataTypeTok{female=}\NormalTok{gender}\OperatorTok{==}\StringTok{'Female'}\NormalTok{,}\DataTypeTok{train=}\KeywordTok{runif}\NormalTok{(}\KeywordTok{nrow}\NormalTok{(.))}\OperatorTok{<}\FloatTok{0.75}\NormalTok{) }\OperatorTok
\StringTok{  }\KeywordTok{filter}\NormalTok{(height }\OperatorTok{>}\StringTok{ }\DecValTok{0} \OperatorTok{&}\StringTok{ }\NormalTok{weight }\OperatorTok{>}\StringTok{ }\DecValTok{0}\NormalTok{);}
\end{Highlighting}
\end{Shaded}

\begin{verbatim}
## Warning in
## tk_tbl.data.frame(data.table::fread("500_Person_Gender_Height_Weight_Index.csv", :
## Warning: No index to preserve. Object otherwise converted to tibble
## successfully.
\end{verbatim}

\begin{Shaded}
\begin{Highlighting}[]
\NormalTok{form <-}\StringTok{ }\NormalTok{female }\OperatorTok{~}\StringTok{ }\NormalTok{height }\OperatorTok{+}
\StringTok{  }\NormalTok{weight }\OperatorTok{+}
\StringTok{  }\KeywordTok{I}\NormalTok{(height}\OperatorTok{^}\DecValTok{2}\NormalTok{) }\OperatorTok{+}
\StringTok{  }\KeywordTok{I}\NormalTok{(weight}\OperatorTok{^}\DecValTok{2}\NormalTok{) }\OperatorTok{+}
\StringTok{  }\NormalTok{height}\OperatorTok{:}\NormalTok{weight;}

\NormalTok{model.gbm <-}\StringTok{ }\KeywordTok{gbm}\NormalTok{(form,}
                 \DataTypeTok{distribution=}\StringTok{"bernoulli"}\NormalTok{,}
\NormalTok{                 info }\OperatorTok\StringTok{ }\KeywordTok{filter}\NormalTok{(train),}
                 \DataTypeTok{n.trees =} \DecValTok{200}\NormalTok{,}
                 \DataTypeTok{interaction.depth =} \DecValTok{5}\NormalTok{,}
                 \DataTypeTok{shrinkage=}\FloatTok{0.1}\NormalTok{);}
\KeywordTok{summary}\NormalTok{(model.gbm,}\DataTypeTok{plot=}\OtherTok{FALSE}\NormalTok{)}
\end{Highlighting}
\end{Shaded}

\begin{verbatim}
##                         var  rel.inf
## weight               weight 53.56935
## height               height 46.43065
## I(height^2)     I(height^2)  0.00000
## I(weight^2)     I(weight^2)  0.00000
## height:weight height:weight  0.00000
\end{verbatim}

\begin{Shaded}
\begin{Highlighting}[]
\NormalTok{test <-}\StringTok{ }\NormalTok{info }\OperatorTok\StringTok{ }\KeywordTok{filter}\NormalTok{(}\OperatorTok{!}\NormalTok{train);}
\NormalTok{test}\OperatorTok{$}\NormalTok{female.p.gbm <-}\StringTok{ }\KeywordTok{predict}\NormalTok{(model.gbm, test, }\DataTypeTok{type=}\StringTok{"response"}\NormalTok{);}
\end{Highlighting}
\end{Shaded}

\begin{verbatim}
## Using 200 trees...
\end{verbatim}

\begin{Shaded}
\begin{Highlighting}[]
\NormalTok{p1 <-}\StringTok{ }\KeywordTok{ggplot}\NormalTok{(test, }\KeywordTok{aes}\NormalTok{(female.p.gbm)) }\OperatorTok{+}\StringTok{ }\KeywordTok{geom_density}\NormalTok{()}

\NormalTok{info1 <-}\StringTok{ }\NormalTok{timetk}\OperatorTok{::}\KeywordTok{tk_tbl}\NormalTok{(data.table}\OperatorTok{::}\KeywordTok{fread}\NormalTok{(}\StringTok{"datasets_26073_33239_weight-height.csv"}\NormalTok{, }\DataTypeTok{header=}\NormalTok{T, }\DataTypeTok{stringsAsFactors=}\NormalTok{T)) }\OperatorTok
\StringTok{  }\KeywordTok{drop_na}\NormalTok{() }\OperatorTok\StringTok{ }
\StringTok{  }\KeywordTok{nice_names}\NormalTok{() }\OperatorTok
\StringTok{  }\KeywordTok{mutate}\NormalTok{(}\DataTypeTok{female=}\NormalTok{gender}\OperatorTok{==}\StringTok{'Female'}\NormalTok{,}\DataTypeTok{train=}\KeywordTok{runif}\NormalTok{(}\KeywordTok{nrow}\NormalTok{(.))}\OperatorTok{<}\FloatTok{0.75}\NormalTok{) }\OperatorTok
\StringTok{  }\KeywordTok{filter}\NormalTok{(height }\OperatorTok{>}\StringTok{ }\DecValTok{0} \OperatorTok{&}\StringTok{ }\NormalTok{weight }\OperatorTok{>}\StringTok{ }\DecValTok{0}\NormalTok{);}
\end{Highlighting}
\end{Shaded}

\begin{verbatim}
## Warning in tk_tbl.data.frame(data.table::fread("datasets_26073_33239_weight-
## height.csv", : Warning: No index to preserve. Object otherwise converted to
## tibble successfully.
\end{verbatim}

\begin{Shaded}
\begin{Highlighting}[]
\NormalTok{model.gbm1 <-}\StringTok{ }\KeywordTok{gbm}\NormalTok{(female }\OperatorTok{~}\StringTok{ }\NormalTok{height }\OperatorTok{+}
\StringTok{                   }\NormalTok{weight }\OperatorTok{+}
\StringTok{                   }\KeywordTok{I}\NormalTok{(height}\OperatorTok{^}\DecValTok{2}\NormalTok{) }\OperatorTok{+}
\StringTok{                   }\KeywordTok{I}\NormalTok{(weight}\OperatorTok{^}\DecValTok{2}\NormalTok{) }\OperatorTok{+}
\StringTok{                   }\NormalTok{height}\OperatorTok{:}\NormalTok{weight,}
                 \DataTypeTok{distribution=}\StringTok{"bernoulli"}\NormalTok{,}
\NormalTok{                 info1 }\OperatorTok\StringTok{ }\KeywordTok{filter}\NormalTok{(train),}
                 \DataTypeTok{n.trees =} \DecValTok{200}\NormalTok{,}
                 \DataTypeTok{interaction.depth =} \DecValTok{5}\NormalTok{,}
                 \DataTypeTok{shrinkage=}\FloatTok{0.1}\NormalTok{);}
\KeywordTok{summary}\NormalTok{(model.gbm1,}\DataTypeTok{plot=}\OtherTok{FALSE}\NormalTok{)}
\end{Highlighting}
\end{Shaded}

\begin{verbatim}
##                         var   rel.inf
## weight               weight 91.704968
## height               height  8.295032
## I(height^2)     I(height^2)  0.000000
## I(weight^2)     I(weight^2)  0.000000
## height:weight height:weight  0.000000
\end{verbatim}

\begin{Shaded}
\begin{Highlighting}[]
\NormalTok{test1 <-}\StringTok{ }\NormalTok{info1 }\OperatorTok\StringTok{ }\KeywordTok{filter}\NormalTok{(}\OperatorTok{!}\NormalTok{train);}
\NormalTok{test1}\OperatorTok{$}\NormalTok{female.p.gbm <-}\StringTok{ }\KeywordTok{predict}\NormalTok{(model.gbm1, test1, }\DataTypeTok{type=}\StringTok{"response"}\NormalTok{);}
\end{Highlighting}
\end{Shaded}

\begin{verbatim}
## Using 200 trees...
\end{verbatim}

\begin{Shaded}
\begin{Highlighting}[]
\NormalTok{p2 <-}\StringTok{ }\KeywordTok{ggplot}\NormalTok{(test1, }\KeywordTok{aes}\NormalTok{(female.p.gbm)) }\OperatorTok{+}\StringTok{ }\KeywordTok{geom_density}\NormalTok{()}

\KeywordTok{grid.arrange}\NormalTok{(p1, p2, }\DataTypeTok{ncol=}\DecValTok{2}\NormalTok{)}
\end{Highlighting}
\end{Shaded}

\includegraphics{HW5_files/figure-latex/unnamed-chunk-1-1.pdf}

\begin{Shaded}
\begin{Highlighting}[]
\KeywordTok{c}\NormalTok{(}\KeywordTok{sum}\NormalTok{((test}\OperatorTok{$}\NormalTok{female.p.gbm}\OperatorTok{>}\FloatTok{0.5}\NormalTok{)}\OperatorTok{==}\NormalTok{test}\OperatorTok{$}\NormalTok{female)}\OperatorTok{/}\KeywordTok{nrow}\NormalTok{(test),}
  \KeywordTok{sum}\NormalTok{(}\OtherTok{FALSE}\OperatorTok{==}\NormalTok{test}\OperatorTok{$}\NormalTok{female)}\OperatorTok{/}\KeywordTok{nrow}\NormalTok{(test),}
  \KeywordTok{sum}\NormalTok{((test1}\OperatorTok{$}\NormalTok{female.p.gbm}\OperatorTok{>}\FloatTok{0.5}\NormalTok{)}\OperatorTok{==}\NormalTok{test1}\OperatorTok{$}\NormalTok{female)}\OperatorTok{/}\KeywordTok{nrow}\NormalTok{(test1),}
  \KeywordTok{sum}\NormalTok{(}\OtherTok{FALSE}\OperatorTok{==}\NormalTok{test1}\OperatorTok{$}\NormalTok{female)}\OperatorTok{/}\KeywordTok{nrow}\NormalTok{(test1));}
\end{Highlighting}
\end{Shaded}

\begin{verbatim}
## [1] 0.5333333 0.4740741 0.9110567 0.4957160
\end{verbatim}

\hypertarget{q2}{%
\section{Q2}\label{q2}}

\begin{enumerate}
\def\labelenumi{\arabic{enumi}.}
\tightlist
\item
  we can see that there are some irregularities with a total of 5, and
  we remove them
\end{enumerate}

\begin{Shaded}
\begin{Highlighting}[]
\NormalTok{info2 <-}\StringTok{ }\NormalTok{timetk}\OperatorTok{::}\KeywordTok{tk_tbl}\NormalTok{(data.table}\OperatorTok{::}\KeywordTok{fread}\NormalTok{(}\StringTok{"datasets_38396_60978_charcters_stats.csv"}\NormalTok{, }\DataTypeTok{header=}\NormalTok{T, }\DataTypeTok{stringsAsFactors=}\NormalTok{T)) }\OperatorTok
\StringTok{  }\KeywordTok{drop_na}\NormalTok{() }\OperatorTok\StringTok{ }
\StringTok{  }\KeywordTok{nice_names}\NormalTok{() }\OperatorTok
\StringTok{  }\KeywordTok{mutate}\NormalTok{(}\DataTypeTok{train=}\KeywordTok{runif}\NormalTok{(}\KeywordTok{nrow}\NormalTok{(.))}\OperatorTok{<}\FloatTok{0.75}\NormalTok{) }
\end{Highlighting}
\end{Shaded}

\begin{verbatim}
## Warning in
## tk_tbl.data.frame(data.table::fread("datasets_38396_60978_charcters_stats.csv", :
## Warning: No index to preserve. Object otherwise converted to tibble
## successfully.
\end{verbatim}

\begin{Shaded}
\begin{Highlighting}[]
\NormalTok{info2}
\end{Highlighting}
\end{Shaded}

\begin{verbatim}
## # A tibble: 611 x 10
##    name  alignment intelligence strength speed durability power combat total
##    <fct> <fct>            <int>    <int> <int>      <int> <int>  <int> <int>
##  1 3-D ~ good                50       31    43         32    25     52   233
##  2 A-Bo~ good                38      100    17         80    17     64   316
##  3 Abe ~ good                88       14    35         42    35     85   299
##  4 Abin~ good                50       90    53         64    84     65   406
##  5 Abom~ bad                 63       80    53         90    55     95   436
##  6 Abra~ bad                 88      100    83         99   100     56   526
##  7 Adam~ good                63       10    12        100    71     64   320
##  8 Adam~ good                 1        1     1          1     0      1     5
##  9 Agen~ good                 1        1     1          1     0      1     5
## 10 Agen~ good                10        8    13          5     5     20    61
## # ... with 601 more rows, and 1 more variable: train <lgl>
\end{verbatim}

\begin{enumerate}
\def\labelenumi{\arabic{enumi}.}
\setcounter{enumi}{1}
\tightlist
\item
  seems we only needs pc1
\end{enumerate}

\begin{Shaded}
\begin{Highlighting}[]
\NormalTok{info2 <-}\StringTok{ }\NormalTok{info2 }\OperatorTok\StringTok{ }\KeywordTok{filter}\NormalTok{(total }\OperatorTok{>}\StringTok{ }\DecValTok{5}\NormalTok{)}

\NormalTok{pcs <-}\StringTok{ }\KeywordTok{prcomp}\NormalTok{(info2[,}\DecValTok{3}\OperatorTok{:}\DecValTok{9}\NormalTok{]);}
\KeywordTok{summary}\NormalTok{(pcs)}
\end{Highlighting}
\end{Shaded}

\begin{verbatim}
## Importance of components:
##                             PC1      PC2      PC3      PC4      PC5      PC6
## Standard deviation     115.6597 24.68437 23.19102 19.01658 17.74925 17.02995
## Proportion of Variance   0.8635  0.03933  0.03472  0.02334  0.02034  0.01872
## Cumulative Proportion    0.8635  0.90288  0.93760  0.96094  0.98128  1.00000
##                              PC7
## Standard deviation     2.551e-14
## Proportion of Variance 0.000e+00
## Cumulative Proportion  1.000e+00
\end{verbatim}

\begin{enumerate}
\def\labelenumi{\arabic{enumi}.}
\setcounter{enumi}{2}
\tightlist
\item
  I think not. The features are already in Percentile. Some of them are
  out of limitation but I think it is still in the scale. Here I use
  durability as an example
\end{enumerate}

\begin{Shaded}
\begin{Highlighting}[]
\KeywordTok{data.frame}\NormalTok{(}\DataTypeTok{min=}\KeywordTok{sapply}\NormalTok{(info2[,}\DecValTok{3}\OperatorTok{:}\DecValTok{10}\NormalTok{],min),}\DataTypeTok{max=}\KeywordTok{sapply}\NormalTok{(info2[,}\DecValTok{3}\OperatorTok{:}\DecValTok{10}\NormalTok{],max))}
\end{Highlighting}
\end{Shaded}

\begin{verbatim}
##              min max
## intelligence   1 113
## strength       1 100
## speed          1 100
## durability     1 120
## power          0 100
## combat         1 101
## total         36 581
## train          0   1
\end{verbatim}

\begin{Shaded}
\begin{Highlighting}[]
\NormalTok{top5 <-}\StringTok{ }\NormalTok{info2 }\OperatorTok\StringTok{ }\KeywordTok{top_n}\NormalTok{(}\DecValTok{5}\NormalTok{,durability)}
\NormalTok{top5[}\KeywordTok{with}\NormalTok{(top5, }\KeywordTok{order}\NormalTok{(}\OperatorTok{-}\NormalTok{durability)), ]}
\end{Highlighting}
\end{Shaded}

\begin{verbatim}
## # A tibble: 55 x 10
##    name  alignment intelligence strength speed durability power combat total
##    <fct> <fct>            <int>    <int> <int>      <int> <int>  <int> <int>
##  1 Doom~ bad                 88       80    67        120   100     90   545
##  2 Star~ good                88       85   100        110   100     85   568
##  3 Nova  good               100       85    67        101   100     85   538
##  4 Silv~ good                63      100    84        101   100     32   480
##  5 Adam~ good                63       10    12        100    71     64   320
##  6 Amazo bad                 75      100   100        100   100    100   575
##  7 Apoc~ bad                100      100    33        100   100     60   493
##  8 Beyo~ good                88      100    23        100   100     56   467
##  9 Biza~ neutral             75       95   100        100    95     85   550
## 10 Blac~ bad                 88      100    92        100    89     56   525
## # ... with 45 more rows, and 1 more variable: train <lgl>
\end{verbatim}

\begin{enumerate}
\def\labelenumi{\arabic{enumi}.}
\setcounter{enumi}{3}
\tightlist
\item
  I create a new column like total and write the result of comparison in
  compare Then show the top 5 of compare and it seems they are equal (no
  value 1 which means they are not equal)
\end{enumerate}

\begin{Shaded}
\begin{Highlighting}[]
\NormalTok{sum <-}\StringTok{ }\KeywordTok{transform}\NormalTok{(info2, }\DataTypeTok{sum=}\KeywordTok{rowSums}\NormalTok{(info2[,}\KeywordTok{c}\NormalTok{(}\DecValTok{7}\NormalTok{,}\DecValTok{3}\NormalTok{,}\DecValTok{4}\NormalTok{,}\DecValTok{5}\NormalTok{,}\DecValTok{6}\NormalTok{,}\DecValTok{8}\NormalTok{)]))}
\NormalTok{sum}\OperatorTok{$}\NormalTok{compare <-}\StringTok{ }\KeywordTok{ifelse}\NormalTok{(sum}\OperatorTok{$}\NormalTok{total }\OperatorTok{==}\StringTok{ }\NormalTok{sum}\OperatorTok{$}\NormalTok{sum, }\DecValTok{0}\NormalTok{, }\DecValTok{1}\NormalTok{)}
\NormalTok{comp <-}\StringTok{ }\NormalTok{sum }\OperatorTok\StringTok{ }\KeywordTok{top_n}\NormalTok{(}\DecValTok{5}\NormalTok{,compare)}
\NormalTok{comp[}\KeywordTok{with}\NormalTok{(comp, }\KeywordTok{order}\NormalTok{(}\OperatorTok{-}\NormalTok{compare)), ]}\OperatorTok{$}\NormalTok{compare}
\end{Highlighting}
\end{Shaded}

\begin{verbatim}
##   [1] 0 0 0 0 0 0 0 0 0 0 0 0 0 0 0 0 0 0 0 0 0 0 0 0 0 0 0 0 0 0 0 0 0 0 0 0 0
##  [38] 0 0 0 0 0 0 0 0 0 0 0 0 0 0 0 0 0 0 0 0 0 0 0 0 0 0 0 0 0 0 0 0 0 0 0 0 0
##  [75] 0 0 0 0 0 0 0 0 0 0 0 0 0 0 0 0 0 0 0 0 0 0 0 0 0 0 0 0 0 0 0 0 0 0 0 0 0
## [112] 0 0 0 0 0 0 0 0 0 0 0 0 0 0 0 0 0 0 0 0 0 0 0 0 0 0 0 0 0 0 0 0 0 0 0 0 0
## [149] 0 0 0 0 0 0 0 0 0 0 0 0 0 0 0 0 0 0 0 0 0 0 0 0 0 0 0 0 0 0 0 0 0 0 0 0 0
## [186] 0 0 0 0 0 0 0 0 0 0 0 0 0 0 0 0 0 0 0 0 0 0 0 0 0 0 0 0 0 0 0 0 0 0 0 0 0
## [223] 0 0 0 0 0 0 0 0 0 0 0 0 0 0 0 0 0 0 0 0 0 0 0 0 0 0 0 0 0 0 0 0 0 0 0 0 0
## [260] 0 0 0 0 0 0 0 0 0 0 0 0 0 0 0 0 0 0 0 0 0 0 0 0 0 0 0 0 0 0 0 0 0 0 0 0 0
## [297] 0 0 0 0 0 0 0 0 0 0 0 0 0 0 0 0 0 0 0 0 0 0 0 0 0 0 0 0 0 0 0 0 0 0 0 0 0
## [334] 0 0 0 0 0 0 0 0 0 0 0 0 0 0 0 0 0 0 0 0 0 0 0 0 0 0 0 0 0 0 0 0 0 0 0 0 0
## [371] 0 0 0 0 0 0 0 0 0 0 0 0 0 0 0 0 0 0 0 0 0 0 0 0 0 0 0 0 0 0 0 0 0 0 0 0 0
## [408] 0 0 0 0 0 0 0 0 0 0 0 0 0 0 0 0 0 0 0 0 0 0 0 0 0 0 0
\end{verbatim}

\begin{enumerate}
\def\labelenumi{\arabic{enumi}.}
\setcounter{enumi}{4}
\tightlist
\item
  No If we check the pca of the data, We can see that PC1 take accounts
  for most of PC1 which has most importance which means ``total'' is the
  principle component of the data, and make other features useless
\end{enumerate}

\begin{Shaded}
\begin{Highlighting}[]
\NormalTok{pcs1 <-}\StringTok{ }\KeywordTok{prcomp}\NormalTok{(info2[,}\DecValTok{3}\OperatorTok{:}\DecValTok{8}\NormalTok{]);}
\KeywordTok{summary}\NormalTok{(pcs1)}
\end{Highlighting}
\end{Shaded}

\begin{verbatim}
## Importance of components:
##                           PC1     PC2     PC3      PC4      PC5      PC6
## Standard deviation     46.664 23.6134 22.8884 18.88294 17.74412 17.02230
## Proportion of Variance  0.516  0.1321  0.1241  0.08449  0.07461  0.06866
## Cumulative Proportion   0.516  0.6481  0.7722  0.85673  0.93134  1.00000
\end{verbatim}

\begin{Shaded}
\begin{Highlighting}[]
\NormalTok{pcs}
\end{Highlighting}
\end{Shaded}

\begin{verbatim}
## Standard deviations (1, .., p=7):
## [1] 1.156597e+02 2.468437e+01 2.319102e+01 1.901658e+01 1.774925e+01
## [6] 1.702995e+01 2.550823e-14
## 
## Rotation (n x k) = (7 x 7):
##                     PC1        PC2        PC3         PC4          PC5
## intelligence 0.08724079 -0.4889535  0.0954139 -0.11144941  0.644722478
## strength     0.22815661  0.4035600 -0.4802491 -0.12337467  0.411166202
## speed        0.12927447  0.1308655  0.1693489  0.88977302 -0.036741381
## durability   0.21580237  0.3804719 -0.1866100 -0.25719501 -0.392768526
## power        0.16116389  0.1036374  0.7651091 -0.33537677 -0.124826239
## combat       0.09590209 -0.6424216 -0.3322937 -0.02450351 -0.493983148
## total        0.91754021 -0.1128403  0.0307192  0.03787365  0.007569386
##                       PC6        PC7
## intelligence  0.416250531 -0.3779645
## strength     -0.476768922 -0.3779645
## speed         0.039751187 -0.3779645
## durability    0.640760245 -0.3779645
## power        -0.327073790 -0.3779645
## combat       -0.283198306 -0.3779645
## total         0.009720944  0.3779645
\end{verbatim}

\begin{Shaded}
\begin{Highlighting}[]
\NormalTok{pcs1}
\end{Highlighting}
\end{Shaded}

\begin{verbatim}
## Standard deviations (1, .., p=6):
## [1] 46.66412 23.61344 22.88843 18.88294 17.74412 17.02230
## 
## Rotation (n x k) = (6 x 6):
##                    PC1         PC2        PC3         PC4          PC5
## intelligence 0.1610814 -0.42123376  0.4180341 -0.04892612  0.654052139
## strength     0.6072148 -0.06014799 -0.4733042 -0.10205240  0.418940908
## speed        0.3083501  0.16436338  0.1023077  0.92906380 -0.009315058
## durability   0.5691860  0.08291554 -0.2156489 -0.22860415 -0.396377363
## power        0.3918877  0.38491920  0.7199693 -0.25647714 -0.115438967
## combat       0.1808714 -0.79805466  0.1609454  0.07740552 -0.475580041
##                      PC6
## intelligence -0.43781780
## strength      0.46661477
## speed        -0.06474356
## durability   -0.64286309
## power         0.31748807
## combat        0.26892912
\end{verbatim}

\begin{enumerate}
\def\labelenumi{\arabic{enumi}.}
\setcounter{enumi}{5}
\tightlist
\item
  I don't know why I can't use the code in lecture 10 so I use another
\end{enumerate}

\begin{Shaded}
\begin{Highlighting}[]
\NormalTok{pca.plot <-}\StringTok{ }\KeywordTok{autoplot}\NormalTok{(pcs, }\DataTypeTok{data =}\NormalTok{ info2, }\DataTypeTok{colour =} \StringTok{'alignment'}\NormalTok{)}
\end{Highlighting}
\end{Shaded}

\begin{verbatim}
## Warning: `select_()` is deprecated as of dplyr 0.7.0.
## Please use `select()` instead.
## This warning is displayed once every 8 hours.
## Call `lifecycle::last_warnings()` to see where this warning was generated.
\end{verbatim}

\begin{Shaded}
\begin{Highlighting}[]
\NormalTok{pca.plot}
\end{Highlighting}
\end{Shaded}

\includegraphics{HW5_files/figure-latex/unnamed-chunk-7-1.pdf}

\hypertarget{q3}{%
\section{Q3}\label{q3}}

Please see the ipynb file attached. The following is the r code use to
plot. I have no findings to the plot.

\begin{Shaded}
\begin{Highlighting}[]
\NormalTok{lowd <-}\StringTok{ }\NormalTok{timetk}\OperatorTok{::}\KeywordTok{tk_tbl}\NormalTok{(data.table}\OperatorTok{::}\KeywordTok{fread}\NormalTok{(}\StringTok{"lowd.csv"}\NormalTok{))}
\end{Highlighting}
\end{Shaded}

\begin{verbatim}
## Warning in tk_tbl.data.frame(data.table::fread("lowd.csv")): Warning: No index
## to preserve. Object otherwise converted to tibble successfully.
\end{verbatim}

\begin{Shaded}
\begin{Highlighting}[]
\NormalTok{(}\KeywordTok{ggplot}\NormalTok{(lowd,}\KeywordTok{aes}\NormalTok{(X1,X2)) }\OperatorTok{+}\StringTok{ }\KeywordTok{geom_point}\NormalTok{(}\KeywordTok{aes}\NormalTok{(}\DataTypeTok{color=}\KeywordTok{factor}\NormalTok{(Alignment))))}
\end{Highlighting}
\end{Shaded}

\includegraphics{HW5_files/figure-latex/unnamed-chunk-8-1.pdf}

\hypertarget{q4}{%
\section{Q4}\label{q4}}

It is in the last part of ipynb attached.

\hypertarget{q5}{%
\section{Q5}\label{q5}}

Theresult is in the picture, in which the final parameters are
``height'', ``weight'' and ``height:weight''.

\begin{Shaded}
\begin{Highlighting}[]
\KeywordTok{library}\NormalTok{(caret)}
\end{Highlighting}
\end{Shaded}

\begin{verbatim}
## Loading required package: lattice
\end{verbatim}

\begin{verbatim}
## 
## Attaching package: 'caret'
\end{verbatim}

\begin{verbatim}
## The following object is masked from 'package:purrr':
## 
##     lift
\end{verbatim}

\begin{Shaded}
\begin{Highlighting}[]
\NormalTok{trainIndex <-}\StringTok{ }\KeywordTok{createDataPartition}\NormalTok{(info}\OperatorTok{$}\NormalTok{female, }\DataTypeTok{p =} \FloatTok{.8}\NormalTok{, }
                                  \DataTypeTok{list =} \OtherTok{FALSE}\NormalTok{, }
                                  \DataTypeTok{times =} \DecValTok{1}\NormalTok{)}
\NormalTok{info}\OperatorTok{$}\NormalTok{female <-}\StringTok{ }\KeywordTok{factor}\NormalTok{(info}\OperatorTok{$}\NormalTok{female);}
\NormalTok{train_ctrl <-}\StringTok{ }\KeywordTok{trainControl}\NormalTok{(}\DataTypeTok{method =} \StringTok{"repeatedcv"}\NormalTok{,}\DataTypeTok{number=}\DecValTok{50}\NormalTok{,}\DataTypeTok{repeats=}\DecValTok{5}\NormalTok{);}
\NormalTok{gbmFit1 <-}\StringTok{ }\KeywordTok{train}\NormalTok{(form, }\DataTypeTok{data =}\NormalTok{ info }\OperatorTok\StringTok{ }\KeywordTok{slice}\NormalTok{(trainIndex), }
                 \DataTypeTok{method =} \StringTok{"gbm"}\NormalTok{, }
                 \DataTypeTok{trControl =}\NormalTok{ train_ctrl,}
                 \DataTypeTok{verbose =} \OtherTok{FALSE}\NormalTok{)}
\KeywordTok{summary}\NormalTok{(gbmFit1)}
\end{Highlighting}
\end{Shaded}

\includegraphics{HW5_files/figure-latex/unnamed-chunk-9-1.pdf}

\begin{verbatim}
##                         var  rel.inf
## height:weight height:weight 41.72747
## height               height 30.91978
## weight               weight 27.35275
## I(height^2)     I(height^2)  0.00000
## I(weight^2)     I(weight^2)  0.00000
\end{verbatim}

\hypertarget{q6}{%
\section{Q6}\label{q6}}

If the size of dataset is unlimited, we can simply divide the dataset
into train set and test set for machine learning. But in practical
situation the dataset is often limited, which means the train set is
likely to bias, because it can only reflect certain parts of the pattern
of the dataset. Because we train the model using the train set, the
model will also likely to bias. This phenomenon will be more obvious if
the size of dataset is more small. K-fold, and other cross validation,
divide the dataset into k group and use 1 for test and others for
training. This method allows to use the whole dataset, instead of some
parts of it, to train the model. And the model will be unbiased based on
the dataset.

Because accuracy cannot fully reflect the performance of model, through
it is an important criterion. For example, if I want to train a model to
classify a dataset which has 990 males and 10 females, a model which
classify all the data as male will have an accuracy of 99\%, but is
useless. We need more approach (for this example, TPR and FPR) to
evaluate the model.

\#Q7

\begin{enumerate}
\def\labelenumi{\arabic{enumi}.}
\item
  Assign a weight to each feature, and then use the predictive model to
  train on these original features.
\item
  After obtaining the weight/ranking of the feature, take the absolute
  value of these values and remove the minimum one.
\item
  Iterate the steps mentioned above until the number of remaining
  features reaches the required limitations.
\end{enumerate}

\end{document}
